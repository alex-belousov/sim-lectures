\documentclass{beamer}

\usepackage{amsfonts}
\usepackage{amsmath}
\usepackage{longtable}
\usepackage{csquotes}
\usepackage{standalone}

\usepackage{graphicx}
\graphicspath{{../pictures/}}

\usepackage{tikz}
\usetikzlibrary{shapes, calc, arrows, decorations.markings,
  decorations.pathmorphing, decorations, patterns, chains, snakes,
  backgrounds, positioning, fit, petri}
\newcommand{\inputpicture}[1]{\input{../drawings/#1}}

\usepackage{listings}
\lstset{language=C, basicstyle=\ttfamily, breaklines=true, keepspaces=true,
  keywordstyle=\color{blue}}

\usepackage{bytefield}

\usefonttheme{professionalfonts}
\usefonttheme{serif}
\usepackage{fontspec}
\setromanfont{CMU Serif}
\setsansfont{CMU Sans Serif}
\setmonofont{CMU Typewriter Text}

\usepackage{hyperref}
\hypersetup{colorlinks=true, linkcolor=black, filecolor=black, citecolor=black,
  urlcolor=blue , pdfauthor=Evgenii Iuliugin <yulyugin@gmail.com>,
  pdftitle=Fundamentals of Full-Platform Simulation}

\usepackage{underscore}
\usepackage{amsthm}

\subtitle{Fundamentals of Full-Platform Simulation}
\subject{Lecture}
\date{\today}

\author[Evgenii Iuliugin]{
  Evgenii Iuliugin \small{\href{mailto:yulyugin@gmail.com}{yulyugin@gmail.com}}}
\typeout{Copyright 2021 Evgenii Iuliugin}

\usetheme{Berlin}
\setbeamertemplate{navigation symbols}{}

\newcommand{\finalslide}{
    {\huge{Thank you!}\par}

    \vfill
    Slides and material are available at
    \url{https://github.com/yulyugin/sim-lectures}
    \vfill

    \tiny{\textit{Note}: All trademarks are the property of their respective
        owners.
        The presented point of view reflects the personal opinion of the author.

        %All the materials are licensed under the Creative Commons
        %Attribution-NonCommercial-ShareAlike 4.0 Worldwide. To view a copy of
        %this license, visit
        %\url{http://creativecommons.org/licenses/by-nc-sa/4.0/}.
    }
}

\title{Programming Languages for Model and Hardware Development}

% TODO: Add SystemRDL and SystemVerilog

\begin{document}

\startslides

\begin{frame}{On the Previous Lecture:}
  \begin{itemize}
    \item Models for different devices --- executable and non-executable.
    \item Functional and cycle-accurate models.
    \item Virtualization and paravirtualization.
  \end{itemize}
\end{frame}

\begin{frame}{Questions:}
  \begin{itemize}
    \item What is a <<magic>> instruction? \pause
    \item What types of devices are typically paravirtualized? \pause
    \item What is device pass-through?
  \end{itemize}
\end{frame}

\section{Simulator Components}

\begin{frame}{Simulator Components}
  \centering
  \inputpicture{breakdown}
\end{frame}

\begin{frame}{Questions}
  \begin{itemize}
    \item What programming language should be used to write a simulator? \pause
    \item What programming language should be used to write models? \pause
    \item The right answer is --- convenient.
  \end{itemize}
\end{frame}

\section{Requirements for Model Development}

\begin{frame}{Tools and Specifications}
  \begin{itemize}
    \item High-level design --- diagrams, queueing models, \dots
    \item Specification --- text, pseudo-code, examples, \dots
    \item Function simulator --- ?
    \item Compiler --- general-purpose programming language.
    \item Disassembler --- ?
    \item Cycle-accurate simulator --- ?
    \item Microcode --- hardware-level instructions.
    \item Register-transfer level (RTL) --- VHDL/Verilog code.
    \item Netlist --- components, connectivity, placement and routing of an
      electric circuit.
  \end{itemize}
  \pause
  \vfill
  \centering
  \textbf{All needs to be consistent!}
\end{frame}

\begin{frame}{General-Purpose Programming Languages}
  \begin{itemize}
    \item Object-oriented programming can be used.
    \item Write a model from scratch every time?
    \item Language specifics:
    \begin{itemize}
      \item int (int32\_t, int64\_t),
      \item endianness,
      \item thread safety,
      \item malloc/free.
    \end{itemize}
  \end{itemize}
\end{frame}

\begin{frame}{Disadvantages}
  \begin{itemize}
    \item General-purpose programming languages provide abstractions suitable
      for wide variety of programs. \pause
    \item Wrong level of abstractions:
    \begin{itemize}
      \item Harder to understand,
      \item Slower to modify and create,
      \item More complex debugging.
    \end{itemize} \pause
    \item Result --- slow development. \pause
    \item Software simulation requires more specific concept --- <<hardware
      abstractions>>.
  \end{itemize}
\end{frame}

\begin{frame}{Hardware Abstractions}
  \begin{itemize}
    \item Signals --- logical level (0, 1),
    \item Bus --- bit groups,
    \item Bit operations,
    \item Transactions --- signal direction,
    \item Extended signal levels --- high-impedance (Z) and unknown (X).
  \end{itemize}
\end{frame}

\begin{frame}{Hardware Abstractions}
  \begin{itemize}
    \item Data storage:
    \begin{itemize}
      \item Register banks,
      \item Channels (FIFO),
      \item Memory banks of different size.
    \end{itemize}
    \item Memory maps --- memory access may be handle by different devices
      depending on address,
    \item Delays --- varies for different actions.
  \end{itemize}
\end{frame}

\begin{frame}{Solution}
  \begin{itemize}
    \item Libraries that implement required abstractions.
    \item Domain Specific Languages (DSL)
  \end{itemize}
  \vfill
  \pause
  Examples:
  \begin{itemize}
    \item Libraries: SystemC/TLM,
    \item Languages: DML, LISA, RTL\dots
  \end{itemize}
\end{frame}

\section{SystemC}

\begin{frame}{SystemC/TLM}
  \begin{tabular}{llp{6cm}}
    Date     & Version & Comments \\
    \hline
    Sep 1999 & 0.9     & First version; Cycled based \\
    Mar 2000 & 1.0     &  Widely accessed major release \\
    Aug 2002 & 2.0     & Add channels \& events; cleaner syntax \\
    Apr 2002 & 2.0.1   & Bug fixes; widely used \\
    Dec 2005 & 2.1.v1  & IEEE approves the IEEE 1666–2005 standard for SystemC \\
    Jun 2008 & 2.2.05  & TLM-2.0.0 library released  \\
    Nov 2011 & 2.3.0   & IEEE approves the IEEE 1666–2011 standard for SystemC \\
  \end{tabular}
\end{frame}

\begin{frame}{SystemC architecture}
  \centering
  \inputpicture{systemc-arch}
\end{frame}

\begin{frame}{SystemC Compilation}
  \centering
  \inputpicture{systemc-compilation}
\end{frame}

% TODO: Time in SystemC

\begin{frame}[fragile]{Example: SystemC code}
  \begin{lstlisting}[basicstyle=\footnotesize]
#include "systemc.h"

SC_MODULE(adder) {        // module (class) declaration
  sc_in<int> a, b;        // ports
  sc_out<int> sum;
  void do_add() {         // process
    sum.write(a.read() + b.read()); //or just sum = a + b
  }
  SC_CTOR(adder) {        // constructor
    SC_METHOD(do_add);    // register do_add to kernel
    sensitive << a << b;  // sensitivity list of do_add
  }
};
  \end{lstlisting}
\end{frame}

\section{Languagues for Device Modeling}

\begin{frame}{DML}
  \begin{itemize}
    \item Used in Simics for device modeling.
    \item Abstractions: registers, register fields, methods set/get,
      discrete events.
    \item Implementation: source-to-source compilation.
  \end{itemize}
\end{frame}

\begin{frame}{DML Compilation}
  \centering
  \inputpicture{dml-compilation}
\end{frame}

% TODO: Use modern DML

\begin{frame}[fragile]{Example: DML code}
  \begin{lstlisting}
dml 1.2;

device simple_dml_device;

parameter desc = "Simple DML device";
parameter documentation = "This is an implementation of simple DML device.";

bank regs {
  parameter register_size = 4;

  register r1 @ 0x1000;
  register r2 @ 0x1004;
}
  \end{lstlisting}
\end{frame}

\begin{frame}{SimGen}
  SimGen~\cite{Larsson-SimGen}:
  \begin{itemize}
    \item Used in Simics for processor modeling.
    \item Abstractions: instructions, opcode fields, mnemonics, semantics.
    \item Implementation: source-to-source compilation.
  \end{itemize}
\end{frame}

\begin{frame}[fragile]{Example: SimGen code}
  \begin{lstlisting}
instruction branch_e_O_A_P(disp22)
  pattern
    op==0 && op2==2 && cond==1 && a==1 && mode==0 && onpage==1
  syntax
    "be,a0x{lx:PC+4*disp22)}"
  semantics
  #{
    if(rCMP_VALUE_A==rCMP_VALUE_B)
      JUMP_REL_ON_PAGE_N(disp22);
    annull_epilogue();
  #}
  \end{lstlisting}
\end{frame}

\begin{frame}{Language for Instruction Set Architecture --- LISA}
  \begin{itemize}
    \item LISA 1.0 is released in 1997~\cite{Zivojnovic-LISA},
    \item LISA 2.0 is part of Synopsis Processor Designer.
  \end{itemize}
  \vfill
  \centering
  \inputpicture{lisa}
\end{frame}

\begin{frame}{Languages for Model Development --- Pros and Cons}
  \begin{itemize}
    \item[$-$] Generated code is not optimal,
    \item[$-$] Generated code is not compact,
    \item[$-$] Learning curve for the new languages.
    \item[$+$] Higher development and modification speed,
    \item[$+$] Consistensy for all components.
  \end{itemize}
\end{frame}

\section{Hardware Development}

% TODO: Cite for Verilog

\begin{frame}{Verilog}
  \begin{itemize}
    \item Created at <<Automated Integrated Design System>> by Phil Moorby \&
      Prabhu Goel in 1984,
    \item Can generate netlist.
    \item Verilog constructs:
    \begin{itemize}
      \item Synthesizable --- can be implemented in hardware,
      \item Non-synthesizable --- for debug and simulation.
    \end{itemize}
  \end{itemize}
\end{frame}

% TODO: Verilog code example

% TODO: Cite for VHDL

\begin{frame}{VHDL}
  \begin{itemize}
    \item Developed in 1983,
    \item Developed at the behest of the U.S. Department of Defense,
    \item Initially aimed for simulation,
    \item Later extended with synthesizable subset.
  \end{itemize}
\end{frame}

% TODO: VHDL code example

\section*{Conclusions}

\begin{frame}[allowframebreaks]{Bibliography}
  \nocite{Hadjiyiannis-ISDL}
  \nocite{Schliebusch-LISA}
  \nocite{Nikhil-BSV}
  \printbibliography
\end{frame}

\begin{frame}{Conclusions}
  \begin{itemize}
    \item Model development should be done using <<convenient>> language.
    \item General-purpose languages don't have right abstractions.
    \item Solution:
    \begin{itemize}
      \item Libraries implementing required abstractions,
      \item Domain specific languages.
    \end{itemize}
  \end{itemize}
\end{frame}

\begin{frame}{On the Next Lecture:}
  Multiprocessor on multiprocessor (MP-on-MP).
\end{frame}

\finalslide

\end{document}
