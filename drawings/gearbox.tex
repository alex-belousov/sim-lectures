\documentclass[tikz,convert={density=150,size=600,outext=.png}]{standalone}
\usetikzlibrary{shapes, calc, arrows, fit, positioning, decorations, patterns, decorations.pathreplacing, chains, snakes}
\input{../setup-web-fonts}
\input{../setup-packages}
\graphicspath{{../pictures/}} % path to pictures, trailing slash is mandatory.

% The actual drawing follows
\begin{document}
    \begin{tikzpicture}[>=latex, font=\small, align=center, inner sep=2pt]

    \node[minimum height = 1.cm] (interp) {Интерпретатор};
    
    \node[minimum height = 1.cm, right = of interp] (bt) {Двоичный\\транслятор};
    
    \node[minimum height = 1.cm, right = of bt] (dex) {Прямое\\исполнение};
    
    \begin{scope}[auto, font=\footnotesize,  text width = 3cm]
    \draw[->] (interp.north) to[bend left] node {Горячий код} ([xshift=-0.1cm] bt.north);
    \draw[->] (bt.north) to[bend left] node {Возможность\\прямого исполнения} (dex.north);
    
    \draw[->] ([xshift=-0.1cm] bt.south) to[bend left] node {Самомодифици-рующийся код} (interp.south);
    \draw[->] (dex.south) to[bend left] node {Частые выходы\\в монитор} (bt.south);
    
    \end{scope}
    \draw[->] ([xshift=0.75cm] dex.south) to[bend left=60] node[auto] {\footnotesize Неподдерживаемые режим или инструкции} ([xshift=-0.75cm] interp.south);
    
    \end{tikzpicture}

\end{document}
